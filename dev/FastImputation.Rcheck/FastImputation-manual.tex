\nonstopmode{}
\documentclass[a4paper]{book}
\usepackage[times,inconsolata,hyper]{Rd}
\usepackage{makeidx}
\usepackage[utf8,latin1]{inputenc}
% \usepackage{graphicx} % @USE GRAPHICX@
\makeindex{}
\begin{document}
\chapter*{}
\begin{center}
{\textbf{\huge Package `FastImputation'}}
\par\bigskip{\large \today}
\end{center}
\begin{description}
\raggedright{}
\item[Type]\AsIs{Package}
\item[Title]\AsIs{Learn from training data then quickly fill in missing data.}
\item[Version]\AsIs{1.2.1}
\item[Date]\AsIs{2016-06-20}
\item[Author]\AsIs{Stephen R. Haptonstahl}
\item[Maintainer]\AsIs{Stephen R. Haptonstahl }\email{srh@haptonstahl.org}\AsIs{}
\item[Suggests]\AsIs{testthat}
\item[Description]\AsIs{TrainFastImputation uses training data to describe a
multivariate normal distribution that the data approximates or
can be transformed into approximating and stores this information
as an object of class FastImputationPatterns. The FastImputation
function uses this FastImputationPatterns object to impute (make
a good guess at) missing data in a single line or a whole dataframe
of data.  This approximates the process used by Amelia
[http://gking.harvard.edu/amelia/] but is much faster when
filling in values for a single line of data.}
\item[License]\AsIs{GPL (>= 2)}
\item[Collate]\AsIs{'FastImputation.R' 'TrainFastImputation.R' 'UnfactorColumns.R'
'BoundNormalizedVariable.R' 'NormalizeBoundedVariable.R'
'LimitToSet.R' 'CovarianceWithMissing.R'}
\item[RoxygenNote]\AsIs{5.0.1}
\item[NeedsCompilation]\AsIs{no}
\end{description}
\Rdcontents{\R{} topics documented:}
\inputencoding{utf8}
\HeaderA{BoundNormalizedVariable}{Take a normalized variable and transform it back to a bounded variable.}{BoundNormalizedVariable}
%
\begin{Description}\relax
This takes variables on the real line and constrains them to be on
a half-line (constrained above or below) or a segment (constrained both
above and below). This is approximately the inverse of 
\code{NormalizeBoundedVariable}; this does not completely reverse the
effect of \code{NormalizeBoundedVariable} because \code{NormalizeBoundedVariable}
first forces values away from the bounds, and this information is lost.
\end{Description}
%
\begin{Usage}
\begin{verbatim}
BoundNormalizedVariable(x, constraints)
\end{verbatim}
\end{Usage}
%
\begin{Arguments}
\begin{ldescription}
\item[\code{x}] A vector, matrix, array, or dataframe with value to be coerced into a range or set.

\item[\code{constraints}] A list of constraints.  See the examples below for formatting details.
\end{ldescription}
\end{Arguments}
%
\begin{Value}
An object of the same class as x with the values transformed into the desired half-line or segment.
\end{Value}
%
\begin{Author}\relax
Stephen R. Haptonstahl \email{srh@haptonstahl.org}
\end{Author}
%
\begin{Examples}
\begin{ExampleCode}
  constraints=list(lower=5)           # lower bound when constrining to an interval
  constraints=list(upper=10)          # upper bound when constraining to an interval
  constraints=list(lower=5, upper=10) # both lower and upper bounds
\end{ExampleCode}
\end{Examples}
\inputencoding{utf8}
\HeaderA{CovarianceWithMissing}{Estimate covariance when data is missing}{CovarianceWithMissing}
%
\begin{Description}\relax
Ignoring missing values can lead to biased estimates of the covariance.
Lounici (2012) gives an unbiased estimator when the data has missing values.
\end{Description}
%
\begin{Usage}
\begin{verbatim}
CovarianceWithMissing(x)
\end{verbatim}
\end{Usage}
%
\begin{Arguments}
\begin{ldescription}
\item[\code{x}] matrix or data.frame, data with each row an observation and each column a variable.
\end{ldescription}
\end{Arguments}
%
\begin{Value}
matrix, unbiased estimate of the covariance.
\end{Value}
%
\begin{Author}\relax
Stephen R. Haptonstahl \email{srh@haptonstahl.org}
\end{Author}
%
\begin{References}\relax
High-dimensional covariance matrix estimation with missing observations.
Karim Lounici. 2012.
\end{References}
\inputencoding{utf8}
\HeaderA{FastImputation}{Use the pattern learned from the training data to impute (fill in good guesses for) missing values.}{FastImputation}
%
\begin{Description}\relax
Like Amelia, FastImputation assumes that the columns of the data are
multivariate normal or can be transformed into approximately
multivariate normal.
\end{Description}
%
\begin{Usage}
\begin{verbatim}
FastImputation(x, patterns, verbose = TRUE)
\end{verbatim}
\end{Usage}
%
\begin{Arguments}
\begin{ldescription}
\item[\code{x}] Vector, matrix, dataframe, or object that can be coerced into a dataframe, possibly with some missing (\code{NA}) values.

\item[\code{patterns}] An object of class 'FastImputationPatterns' generated by \code{TrainFastImputation}.

\item[\code{verbose}] If TRUE then the progress in imputing the data will be shown.
\end{ldescription}
\end{Arguments}
%
\begin{Value}
An object of class 'FastImputationPatterns' that contains
information needed later to impute on a single row.
\end{Value}
%
\begin{Author}\relax
Stephen R. Haptonstahl \email{srh@haptonstahl.org}
\end{Author}
%
\begin{References}\relax
\url{http://gking.harvard.edu/amelia/}
\end{References}
%
\begin{SeeAlso}\relax
\code{\LinkA{TrainFastImputation}{TrainFastImputation}}
\end{SeeAlso}
%
\begin{Examples}
\begin{ExampleCode}
data(FItrain)   # provides FItrain dataset
patterns <- TrainFastImputation(FItrain)

data(FItest)
FItest          # note there is missing data
imputed.data <- FastImputation(FItest, patterns)
imputed.data    # good guesses for missing values are filled in

data(FItrue)
imputation.rmse <- sqrt(sum( (imputed.data - FItrue)^2 )/sum(is.na(FItest)))
imputation.rmse
\end{ExampleCode}
\end{Examples}
\inputencoding{utf8}
\HeaderA{FItest}{Fraud Imputation Test Data}{FItest}
\keyword{datasets}{FItest}
%
\begin{Description}\relax
Observations of Web financial transactions with some 
cells missing. Used with FastImputation.
\end{Description}
%
\begin{Usage}
\begin{verbatim}
FItest
\end{verbatim}
\end{Usage}
%
\begin{Format}
 A data frame with 10 variables and 10000 observations. \begin{enumerate}

\item \code{cust.id}: Internal customer identification number
\item \code{order.id}: Unique identification number for this transaction (row)
\item \code{is.fraud}: 1 if the transaction is fraudulent, 0 otherwise
\item \code{customer.age.yrs}: Customer age in years; may be a decimal
\item \code{spent.days.0to2}: Amount spent in dollars by customer between 0 and 2 days before the current transaction
\item \code{spent.days.3to10}: Amount spent in dollars by customer between 3 and 10 days before the current transaction
\item \code{spent.days.11to30}: Amount spent in dollars by customer between 11 and 30 days before the current transaction
\item \code{geo.ip.fraud.rate}: Fraction between 0 and 1 of transactions from that geographic location (identified by IP address) that have been fraudulent
\item \code{account.age.days}: Integer number of days the customer has had the account
\item \code{days.to.first.purchase} Integer number of days between account creation and the first purchase by the customer

\end{enumerate}
\end{Format}
%
\begin{Author}\relax
Stephen R. Haptonstahl \email{srh@haptonstahl.org}
\end{Author}
%
\begin{Source}\relax
This is simulated data generated to be similar to real data.
\end{Source}
\inputencoding{utf8}
\HeaderA{FItrain}{Fraud Training Data}{FItrain}
\keyword{datasets}{FItrain}
%
\begin{Description}\relax
Complete observations of Web financial transactions.  
Used with TrainFastImputation to prepare for imputing individual
transactions as they come in.
\end{Description}
%
\begin{Usage}
\begin{verbatim}
FItrain
\end{verbatim}
\end{Usage}
%
\begin{Format}
 A data frame with 10 variables and 10000 observations. \begin{enumerate}

\item \code{cust.id}: Internal customer identification number
\item \code{order.id}: Unique identification number for this transaction (row)
\item \code{is.fraud}: 1 if the transaction is fraudulent, 0 otherwise
\item \code{customer.age.yrs}: Customer age in years; may be a decimal
\item \code{spent.days.0to2}: Amount spent in dollars by customer between 0 and 2 days before the current transaction
\item \code{spent.days.3to10}: Amount spent in dollars by customer between 3 and 10 days before the current transaction
\item \code{spent.days.11to30}: Amount spent in dollars by customer between 11 and 30 days before the current transaction
\item \code{geo.ip.fraud.rate}: Fraction between 0 and 1 of transactions from that geographic location (identified by IP address) that have been fraudulent
\item \code{account.age.days}: Integer number of days the customer has had the account
\item \code{days.to.first.purchase} Integer number of days between account creation and the first purchase by the customer

\end{enumerate}
\end{Format}
%
\begin{Author}\relax
Stephen R. Haptonstahl \email{srh@haptonstahl.org}
\end{Author}
%
\begin{Source}\relax
This is simulated data generated to be similar to real data.
\end{Source}
\inputencoding{utf8}
\HeaderA{FItrue}{Fraud "True" Data}{FItrue}
\keyword{datasets}{FItrue}
%
\begin{Description}\relax
Complete observations of Web financial transactions.  
Used to gauge the accuracy of imputation of FItest.
\end{Description}
%
\begin{Usage}
\begin{verbatim}
FItrue
\end{verbatim}
\end{Usage}
%
\begin{Format}
 A data frame with 10 variables and 10000 observations. \begin{enumerate}

\item \code{cust.id}: Internal customer identification number
\item \code{order.id}: Unique identification number for this transaction (row)
\item \code{is.fraud}: 1 if the transaction is fraudulent, 0 otherwise
\item \code{customer.age.yrs}: Customer age in years; may be a decimal
\item \code{spent.days.0to2}: Amount spent in dollars by customer between 0 and 2 days before the current transaction
\item \code{spent.days.3to10}: Amount spent in dollars by customer between 3 and 10 days before the current transaction
\item \code{spent.days.11to30}: Amount spent in dollars by customer between 11 and 30 days before the current transaction
\item \code{geo.ip.fraud.rate}: Fraction between 0 and 1 of transactions from that geographic location (identified by IP address) that have been fraudulent
\item \code{account.age.days}: Integer number of days the customer has had the account
\item \code{days.to.first.purchase} Integer number of days between account creation and the first purchase by the customer

\end{enumerate}
\end{Format}
%
\begin{Author}\relax
Stephen R. Haptonstahl \email{srh@haptonstahl.org}
\end{Author}
%
\begin{Source}\relax
This is simulated data generated to be similar to real data.
\end{Source}
\inputencoding{utf8}
\HeaderA{LimitToSet}{Coerce numeric values into a given set.}{LimitToSet}
%
\begin{Description}\relax
Given some values \code{x} and a set of values \code{set},
each value in \code{x} is changed to the value in \code{set} that
is closest.
\end{Description}
%
\begin{Usage}
\begin{verbatim}
LimitToSet(x, set)
\end{verbatim}
\end{Usage}
%
\begin{Arguments}
\begin{ldescription}
\item[\code{x}] A vector, matrix, array, or dataframe with value to be coerced into a range or set.

\item[\code{set}] A list of values that x will be forced to take on.
\end{ldescription}
\end{Arguments}
%
\begin{Value}
An object of the same class as \code{x} with values replaced as needed to satisfy the constraints.
\end{Value}
%
\begin{Author}\relax
Stephen R. Haptonstahl \email{srh@haptonstahl.org}
\end{Author}
%
\begin{Examples}
\begin{ExampleCode}
x <- runif(100, min=0, max=10)
y <- LimitToSet(x, set=c(1:10))
plot(x, y)

\end{ExampleCode}
\end{Examples}
\inputencoding{utf8}
\HeaderA{NormalizeBoundedVariable}{Take a variable bounded above/below/both and return an unbounded (normalized) variable.}{NormalizeBoundedVariable}
%
\begin{Description}\relax
This transforms bounded variables so that they are not bounded.
First variables are coerced away from the boundaries. by a distance of \code{tol}.
The natural log is used for variables bounded either above or below but not both.
The inverse of the standard normal cumulative distribution function 
(the quantile function) is used for variables bounded above and below.
\end{Description}
%
\begin{Usage}
\begin{verbatim}
NormalizeBoundedVariable(x, constraints, tol = pnorm(-5), trim = TRUE)
\end{verbatim}
\end{Usage}
%
\begin{Arguments}
\begin{ldescription}
\item[\code{x}] A vector, matrix, array, or dataframe with value to be 
coerced into a range or set.

\item[\code{constraints}] A list of constraints.  See the examples below 
for formatting details.

\item[\code{tol}] Variables will be forced to be at least this far away 
from the boundaries.

\item[\code{trim}] If TRUE values in x < lower and values in x > upper 
will be set to lower and upper, respectively, before normalizing.
\end{ldescription}
\end{Arguments}
%
\begin{Value}
An object of the same class as \code{x} with the values 
transformed so that they spread out over any part of the real 
line.

A variable \code{x} that is bounded below by \code{lower} is
transformed to \code{log(x - lower)}.

A variable \code{x} that is bounded above by \code{upper} is
transformed to \code{log(upper - x)}.

A variable \code{x} that is bounded below by \code{lower} and
above by \code{upper} is transformed to 
\code{qnorm((x-lower)/(upper - lower))}.
\end{Value}
%
\begin{Author}\relax
Stephen R. Haptonstahl \email{srh@haptonstahl.org}
\end{Author}
%
\begin{Examples}
\begin{ExampleCode}
  constraints=list(lower=5)           # lower bound when constrining to an interval
  constraints=list(upper=10)          # upper bound when constraining to an interval
  constraints=list(lower=5, upper=10) # both lower and upper bounds
\end{ExampleCode}
\end{Examples}
\inputencoding{utf8}
\HeaderA{TrainFastImputation}{Learn from the training data so that later you can fill in missing data}{TrainFastImputation}
%
\begin{Description}\relax
Like Amelia, FastImputation assumes that the columns of the data are
multivariate normal or can be transformed into approximately
multivariate normal.
\end{Description}
%
\begin{Usage}
\begin{verbatim}
TrainFastImputation(x, constraints = list())
\end{verbatim}
\end{Usage}
%
\begin{Arguments}
\begin{ldescription}
\item[\code{x}] Dataframe containing training data. Can have incomplete rows.

\item[\code{constraints}] A list of constraints.  See the examples below for formatting details.
\end{ldescription}
\end{Arguments}
%
\begin{Value}
An object of class 'FastImputationPatterns' that contains
information needed later to impute on a single row.
\end{Value}
%
\begin{Author}\relax
Stephen R. Haptonstahl \email{srh@haptonstahl.org}
\end{Author}
%
\begin{References}\relax
\url{http://gking.harvard.edu/amelia/}
\end{References}
%
\begin{SeeAlso}\relax
\code{\LinkA{FastImputation}{FastImputation}}
\end{SeeAlso}
%
\begin{Examples}
\begin{ExampleCode}

data(FItrain)   # provides FItrain dataset
patterns <- TrainFastImputation(FItrain)

patterns.with.constraints <- TrainFastImputation(
  FItrain,
  constraints=list(list(1, list(set=0:1)),
                   list(2, list(lower=0)),
                   list(3, list(lower=0)),
                   list(4, list(lower=0)),
                   list(5, list(lower=0)),
                   list(6, list(lower=0, upper=1)),
                   list(7, list(lower=0)),
                   list(8, list(lower=0))))
\end{ExampleCode}
\end{Examples}
\inputencoding{utf8}
\HeaderA{UnfactorColumns}{Convert columns of a dataframe from factors to character or numeric.}{UnfactorColumns}
%
\begin{Description}\relax
Convert columns of a dataframe from factors to character or numeric.
\end{Description}
%
\begin{Usage}
\begin{verbatim}
UnfactorColumns(x)
\end{verbatim}
\end{Usage}
%
\begin{Arguments}
\begin{ldescription}
\item[\code{x}] A dataframe
\end{ldescription}
\end{Arguments}
%
\begin{Value}
A dataframe containing the same data but any \code{factor} columns have been replaced with numeric or character columns.
\end{Value}
%
\begin{Author}\relax
Stephen R. Haptonstahl \email{srh@haptonstahl.org}
\end{Author}
\printindex{}
\end{document}
